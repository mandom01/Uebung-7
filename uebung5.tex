\documentclass{article}
\usepackage{float}
\begin{document}
\title{CS102 \"Ubung 6}
\author{Dominik Mangold}
\maketitle
\section{Ich war hier}
Hollibolli.
\section{First section}
The first section contains this text.
\section{Table}
Fancy things are written in this table.
\begin{table}[h!]
\begin{center}
\begin{tabular}{|c|l|}\hline
12345 & qwertyuiop\\\hline
67890 & asdfghjkl\\\hline
11 & zxcvbnm\\\hline
\end{tabular}
\caption{table example}
\end{center}
\end{table}
\section{Formulas}
\subsection{Pythagorean theorem}
The theorem can be written as an equation relating the lengths of the sides a, b and c:  \[ a^2 + b^2 = c^2 \]
With this formula you can calculate the length of the hypothenuse c: \[c = \sqrt{a^2 + b^2}\]
\subsection{Summation}
The formula for sums is defined by: \[s = \sum_{i = 1}^n i = \frac{n * (n + 1)}{2}\]
\end{document}
